\subsection{Retrospektiv påføring av kontrollerte emneord}

Having taken control over the vocabulary, we made the bold decision of retrospectively adding subject headings to records. With help from the library system providers, BIBSYS, approximately 80000 records were furnished with subject headings in this way. This was possible because we had preserved the connection between each word and the record id(s) it came from. At this stage, we still had one list for each of the 9 ?? department libraries. Accordingly, a word accepted in one list could be omitted in another. This in itself was not wrong, because the different departments needed different subject headings. 

We now proceeded to melt the 9 lists into one. The work of tidying up had only just begun. We found that we had many undetected synonyms being used, as well as spelling mistakes and other errors. We split the vocabulary between us and proofread a chunk of about 3000 headings each. That let us correct many mistakes, but still left some of the synonyms co-existing in the list. Making corrections after the retrospective indexing was done meant that corrections must be done both in the vocabulary itself and in the catalog. We were given access to a command in the library system, which let us make global changes. 

