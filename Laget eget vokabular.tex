\subsection{Laget eget vokabular}
The process of creating a controlled vocabulary was based on re-using the keywords we already had. Using the catalog data we had record ids coupled to keywords and department library holdings. 

A main idea was to exploit this connection. The process was as follows: Each of our department libraries was provided with a list of all keywords that appeared on their catalog records. Due to the nature of the BIBSYS catalog, which is a common catalog used by the vast majority of academic and research libraries in Norway, there was no way of knowing whether the keywords had originally been assigned by us, or by any other library sharing the record in question. The lists varied in length, from a couple of thousand to XXX. They were arranged alphabetically. The task was now to assign a code to each word, stating whether the word should be part of the controlled vocabulary or ignored. Other possibilities were to mark the words as "start of phrase", "latin", "genre", "time", "place", "acronym", or "synonym" to another word in the list. During this coding, the keywords were not seen in context to records, the task being only to consider whether or not each keyword should belong to our controlled vocabulary. 

The coding "start of phrase" was used for all adjectives. In the next step, we harvested the word immediately following this, from each of the records where this start of phrase was used. This resulted in new lists to be checked, and the securing of many useful phrases. 