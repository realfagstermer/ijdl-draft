\subsection{Laget strenger/innførte prekoordinering}
When the subject heading were copyed back into the records and our own inhouse software was ready, our staff started using our controlled vocabulary when assigning subject headings. We used a marcfield fit for making subject strings, and the indexer could now build strings, using elements from the categories subject content headings, time headings, place headings and genre headings. Slowly, the amount of strings grew. 

When in the beginning of this process we assigned categories to subject headings, that meant we could theoretically create subject strings consisting of heading - time - place - genre/bibliographic form. We decided it would be to random to create strings automatically at this point. But we wanted to keep not only the regular subject headings, but also the times, places and genres, so the solution was to copy these also back to the records retrospectively. We did that by repeating the marc field for each subject heading, using only the appropriate subfield. This seemed clever but soon backfired on us. As BIBSYS, our library system provider, started their road to changing from a system they had themselves built over many years, to an international system, they also started preparing to migrate from BIBSYS-marc to Marc21. The field we used (and still use) in BIBSYS-marc would be converted into Marc21 field 650. Both fields accomodate subject strings, but in BIBSYS there is no actual validation in this field, so whereas we have made routines to minify (hæ??) the risk of spelling errors, the system itself never stopped us from using the field without the dollar a subfield. But now we were found out, and told to take action lest we lose the data in all fields with an empty dollar a. 

Lucky for us BIBSYS' first attempt to migrate into a new library system, ended in a mutual agreement on leaving the contract. That bought us time to think. We let this desicion mature over time, testing different scenarios. The easiest option was of course to lose all data not in fields with dollar a subfield. Looking into that, we felt there was so much valuable subject information we had already worked to save, that we would not give up. We needed a way to keep it. Inspired by the work done by Library of Congress to simplify subject strings by breaking them down into facets (FAST), and the corresponding work going on in Sweden, we finally decided to go the same way: To break the string structure down into facets. We would keep the possibility of creating strings consisting of "topical term - general subdivision", and then use headings from the categories time, place and genre as individual facets, moved to appropriate marc fields. 

This was no small task. The inhouse software had to be reprogrammed to fit the new pattern. Simultaneously we decided to move the vocabulary editing function out of Roald, and isolate it. 