\section{Introduction}

(Innledning om historie, emnerapporten 2010, bakgrunn og grunnlag for...)

In 2010 at least X different classification schemes and various controlled and uncontrolled vocabularies were in use at the University of Oslo Library. An internal report suggested a reduction in the number of systems, and the establishment of crosswalks between the remaing systems.

Particuraly grave was the subject indexing situation at the science libraries, were only uncontrolled keywords were used. A new controlled vocabulary was in the planning though, and the merge of the nine former department libraries into a new Science Library created some policy space to realize the plans.

\section{The birth of a new vocabulary}

How do you create a new, controlled vocabulary that can be still be used to search all the literature you already have?
We decided to base the new vocabulary on the keywords we already had. Using the catalog data we had record ids coupled to keywords and department library holdings. 

A main idea was to exploit this connection. The process was as follows: Each of our department libraries was provided with a list of all keywords that appeared on their catalog records. Due to the nature of the BIBSYS catalog, which is a common catalog used by the vast majority of academic and research libraries in Norway, there was no way of knowing whether the keywords had originally been assigned by us, or by any other library sharing the record in question. The lists varied in length, from a couple of thousand to XXX. They were arranged alphabetically. The task was now to assign a code to each word, stating whether the word should be part of the controlled vocabulary or ignored. Other possibilities were to mark the words as "start of phrase", "latin", "genre", "time", "place", "acronym", or "synonym" to another word in the list. During this coding, the keywords were not seen in context to records, the task being only to consider whether or not each keyword should belong to our controlled vocabulary. 

The coding "start of phrase" was used for all adjectives. In the next step, we harvested the word immediately following this, from each of the records where this start of phrase was used. This resulted in new lists to be checked, and the securing of many useful phrases. 

\section{Retrospektiv påføring av kontrollerte emneord}

Having taken control over the vocabulary, we made the bold decision of retrospectively adding subject headings to records. With help from the library system providers, BIBSYS, approximately 80000 records were furnished with subject headings in this way. This was possible because we had preserved the connection between each word and the record id(s) it came from. At this stage, we still had one list for each of the 9 ?? department libraries. Accordingly, a word accepted in one list could be omitted in another. This in itself was not wrong, because the different departments needed different subject headings. 

We now proceeded to melt the 9 lists into one. The work of tidying up had only just begun. We found that we had many undetected synonyms being used, as well as spelling mistakes and other errors. We split the vocabulary between us and proofread a chunk of about 3000 headings each. That let us correct many mistakes, but still left some of the synonyms co-existing in the list. Making corrections after the retrospective indexing was done meant that corrections must be done both in the vocabulary itself and in the catalog. We were given access to a command in the library system, which let us make global changes. 

\section{Laget strenger/innførte prekoordinering}

When the subject heading were copyed back into the records and our own inhouse software was ready, our staff started using our controlled vocabulary when assigning subject headings. We used a marcfield fit for making subject strings, and the indexer could now build strings, using elements from the categories subject content headings, time headings, place headings and genre headings. Slowly, the amount of strings grew. 

When in the beginning of this process we assigned categories to subject headings, that meant we could theoretically create subject strings consisting of heading - time - place - genre/bibliographic form. We decided it would be to random to create strings automatically at this point. But we wanted to keep not only the regular subject headings, but also the times, places and genres, so the solution was to copy these also back to the records retrospectively. We did that by repeating the marc field for each subject heading, using only the appropriate subfield. This seemed clever but soon backfired on us. As BIBSYS, our library system provider, started their road to changing from a system they had themselves built over many years, to an international system, they also started preparing to migrate from BIBSYS-marc to Marc21. The field we used (and still use) in BIBSYS-marc would be converted into Marc21 field 650. Both fields accomodate subject strings, but in BIBSYS there is no actual validation in this field, so whereas we have made routines to minify (hæ??) the risk of spelling errors, the system itself never stopped us from using the field without the dollar a subfield. But now we were found out, and told to take action lest we lose the data in all fields with an empty dollar a. 

Lucky for us BIBSYS' first attempt to migrate into a new library system, ended in a mutual agreement on leaving the contract. That bought us time to think. We let this desicion mature over time, testing different scenarios. The easiest option was of course to lose all data not in fields with dollar a subfield. Looking into that, we felt there was so much valuable subject information we had already worked to save, that we would not give up. We needed a way to keep it. Inspired by the work done by Library of Congress to simplify subject strings by breaking them down into facets (FAST), and the corresponding work going on in Sweden, we finally decided to go the same way: To break the string structure down into facets. We would keep the possibility of creating strings consisting of "topical term - general subdivision", and then use headings from the categories time, place and genre as individual facets, moved to appropriate marc fields. 

This was no small task. The inhouse software had to be reprogrammed to fit the new pattern. Simultaneously we decided to move the vocabulary editing function out of Roald, and isolate it. 

\section{Resten}

Laget eget vokabular

Retrospektiv påføring av kontrollerte emneord

Laget strenger/innførte prekoordinering

Laget et emnesøk (practical applications)

Revisjon av emneordsstrukturen (fra 687 til flere felter)

Flerspråklighet. Engelsk, nynorsk, latin (vitenskapelige navn). Forkortelser, msc, 
synonymer, se også-henvisninger. 

Laget mappingverktøy

Mappet vokabularet mot tekord, dewey

Laget redigeringsprogam 

Laget brukerverktøy for emneordsetting

Publisert lenkede data

Planer framover/daglig drift/samarbeid (varige/ongoing prosesser)